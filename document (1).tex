\documentclass{article}
\usepackage[utf8]{inputenc}
\usepackage{indentfirst}
\usepackage{inputenc}
\usepackage[brazilian]{babel}
\usepackage{graphicx}
%para deixar as palavras em português
\usepackage[brazilian]{babel}
%margens da página
\usepackage[margin=2.5cm]{geometry}
%para definir espacamentos
\usepackage{setspace}
%numerar e colocar a referência no súmario
\usepackage[nottoc,numbib]{tocbibind}
% para criar tabela que pode ser quebrado em várias páginas
\usepackage{longtable} 
% para linhas duplas na tabela
\usepackage{hhline}
% para elaborar preâmbulo do documento
\usepackage{fancyhdr}
\pagestyle{fancy}
% para carregar a última página
\usepackage{lastpage}





\begin{document}
	\begin{center}
	\large
			\textbf{INSTITUTO FEDERAL GOIANO\\
				NÚCLEO DE INFORMÁTICA\\
				Sistemas De Informação\\
				Estrutura de Dados II\\\vspace{4.5cm}
			%	Organizze: \\\vspace{0.5cm}
				ALGORITMOS DE ORDENAÇÃO \\\vspace{3.0cm}
			  % \\\vspace{3.5cm}
				\textbf{Giovani Barbosa dos Santos Filho}\\\vspace{2cm}
				PROFESSOR: Junio Cesar de Lima \\\vspace{8cm}
				\today
			}
	\end{center}
	 

	 \newpage

	 
	 
	 \section{Resultado dos Algoritmos}
	 	
	\indent Utilizando os mesmos vetores de tamanho 10, 20, 30, 40, 50 em todos os algoritmos, obtemos os seguintes resultados:

	\begin{center}
		\vspace{0.5cm}
		%\centering % coloca a tabela no centro
		\begin{tabular}[htbp]{| l |c |c |c |c |c  |}%\ quantidade de linhas , c = centralizado, r = direita, l = esquerda
			\hline 
			Tipo de Algoritmo& 10&20&30&40&50\\
			\hline %\colocar linha entre as colunas
			BubbleSort: &9826,52 &15608,22&23109,04&34784,58& 75261,75\\ %\ '&' serve para separar os elementos
			\hline
		 %\colocar linha entre as colunas
			InsertionSort: &2078,81 &3780,35&5375,10&7187,51& 8496,21\\ %\ '&' serve para separar os elementos
			\hline	
			%\colocar linha entre as colunas
			ShellSort: &2375,66 &5226,38&6588,30&10704,67& 10369,92\\ %\ '&' serve para separar os elementos
			\hline	
			%\colocar linha entre as colunas
			QuickSort: &4338,90 &6854,79&12756,48&17888,06&8818,57 \\ %\ '&' serve para separar os elementos
			\hline	
			%\colocar linha entre as colunas
			MergeSort: & 6482,69 &11193,65&17462,03&10750,61& 11485,70 \\ %\ '&' serve para separar os elementos
			\hline		
			%\colocar linha entre as colunas
			HeapSort: & 5942,10&14926,53&22821,00&10370,44&11631,10 \\ %\ '&' serve para separar os elementos
			\hline
			
			
		\end{tabular}
		
		Tempo em Nanossegundos
	
\end{center}	

		

	
\vspace{0.5cm}
	\indent Utilizando os mesmos vetores de tamanho 60, 70, 80, 90, 100, em todos os algoritmos, obtemos os seguintes resultados:

\begin{center}
	\vspace{0.5cm}
	%\centering % coloca a tabela no centro
	\begin{tabular}[htbp]{| l |c |c |c |c |c  |}%\ quantidade de linhas , c = centralizado, r = direita, l = esquerda
		\hline 
		Tipo de Algoritmo& 60&70&80&90&100\\
		\hline %\colocar linha entre as colunas
		BubbleSort: &37161,85 &49114,96&46469,98& 51610,09&70091,25
		 \\ %\ '&' serve para separar os elementos
		\hline
		%\colocar linha entre as colunas
		InsertionSort: &10234,75 &9779,66&9362,61&10649,60& 13126,42\\ %\ '&' serve para separar os elementos
		\hline	
		%\colocar linha entre as colunas
		ShellSort: &15329,19 & 15657,00&16249,90&18486,31&22888,64 \\ %\ '&' serve para separar os elementos
		\hline	
		%\colocar linha entre as colunas
		QuickSort: &17365,28 &11011,42&10578,00&11906,87& 13646,53\\ %\ '&' serve para separar os elementos
		\hline	
		%\colocar linha entre as colunas
		MergeSort: & 36691,31&14557,40&19044,43&15410,12&69828,19\\ %\ '&' serve para separar os elementos
		\hline		
		%\colocar linha entre as colunas
		HeapSort: &10994,42 &14651,82&15302,29&16057,70& 27978,94\\ %\ '&' serve para separar os elementos
		\hline
		
		
	\end{tabular}
		
			Tempo em Nanossegundos
		
\end{center}	
	
\vspace{0.5cm}
 \indent Utilizando os mesmos vetores de tamanho  1000, 2000, 3000, 4000, 5000 em todos os algoritmos, obtemos os seguintes resultados:

\begin{center}
	\vspace{0.5cm}
	%\centering % coloca a tabela no centro
	\begin{tabular}[htbp]{| l |c |c |c |c |c  |}%\ quantidade de linhas , c = centralizado, r = direita, l = esquerda
		\hline 
		Tipo de Algoritmo& 1000&2000&3000&4000&5000\\
		\hline %\colocar linha entre as colunas
		BubbleSort: &418137,81 &1392992,00&2652620,75&4373003,50&6794470,50 \\ %\ '&' serve para separar os elementos
		\hline
		%\colocar linha entre as colunas
		InsertionSort: & 53563,81&83199,94&120131,53&150856,41&171596,58 \\ %\ '&' serve para separar os elementos
		\hline	
		%\colocar linha entre as colunas
		ShellSort: &96100,12 &110454,14& 128081,72& 142852,89&174796,36\\ %\ '&' serve para separar os elementos
		\hline	
		%\colocar linha entre as colunas
		QuickSort: &147398,69 &370552,84&736617,75&1368031,25&1727190,25 \\ %\ '&' serve para separar os elementos
		\hline	
		%\colocar linha entre as colunas
		MergeSort: &101351,51 &157899,80&219977,98&289794,53&308359,53 \\ %\ '&' serve para separar os elementos
		\hline		
		%\colocar linha entre as colunas
		HeapSort: &109002,31 &180358,80&302585,78&297036,91& 348612,16\\ %\ '&' serve para separar os elementos
		\hline
		
		
	\end{tabular}
	
		Tempo em Nanossegundos
\end{center}
\vspace{0.5cm}	

\indent Utilizando os mesmos vetores de tamanho  6000, 7000, 8000, 9000, 10000 em todos os algoritmos, obtemos os seguintes resultados:

	

%\newpage
\begin{center}
	\vspace{0.5cm}
	%\centering % coloca a tabela no centro
\begin{tabular}[htbp]{| l |c |c |c |c |c  |}%\ quantidade de linhas , c = centralizado, r = direita, l = esquerda
		\hline 
		Tipo de Algoritmo& 6000&7000&8000&9000&10000\\
		\hline %\colocar linha entre as colunas
		BubbleSort: &9733176,00 &13214157,00&15144973,00&18861112,00& 23444644,00\\ %\ '&' serve para separar os elementos
		\hline
		%\colocar linha entre as colunas
		InsertionSort: &190591,05 &242790,50&267367,53&300153,94&330128,03 \\ %\ '&' serve para separar os elementos
		\hline	
		%\colocar linha entre as colunas
		ShellSort: &195560,36 &252773,36&283383,25&315359,78& 349154,41\\ %\ '&' serve para separar os elementos
		\hline	
		%\colocar linha entre as colunas
		QuickSort: &2658867,75 &3181924,75&4754011,00&5370520,50& 6236808,50 \\ %\ '&' serve para separar os elementos
		\hline	
		%\colocar linha entre as colunas
		MergeSort: &352911,03 &419205,56&464939,38&531773,13&586198,38\\ %\ '&' serve para separar os elementos
		\hline		
		%\colocar linha entre as colunas
		HeapSort: &433164,72 & 478246,38&533417,06&602822,75& 660001,75\\ %\ '&' serve para separar os elementos
		\hline
		
		
	\end{tabular}
	
	Tempo em Nanossegundos
\end{center}

\newpage

\section{Gráficos de Tempo}

\begin{figure}[h] %\h   significa here = aqui, e [ht] para se a imagem não couber no resto da folha
	
	\includegraphics[scale=0.7]{1050.png} 
	 %\Legenda %\para figuras
\end{figure}%\

\begin{figure}[h] %\h   significa here = aqui, e [ht] para se a imagem não couber no resto da folha
	
	\includegraphics[scale=0.7]{60100.png} 
	%\Legenda %\para figuras
\end{figure}%\

\newpage

\begin{figure}[h] %\h   significa here = aqui, e [ht] para se a imagem não couber no resto da folha
	
	\includegraphics[scale=0.7]{10005000.png} 
	%\Legenda %\para figuras
\end{figure}%\
\begin{figure}[h] %\h   significa here = aqui, e [ht] para se a imagem não couber no resto da folha
	
	\includegraphics[scale=0.7]{600010000.png} 
	%\Legenda %\para figuras
\end{figure}%\


\newpage
\section{Conclusões}
\vspace{0.5cm}

	 Como podemos ver nos gráficos, a medida em que o número de elementos do vetor aumenta, o tempo de execução também aumenta. O algoritmo Bubble e o Selection são os mais demorados quando se trata de vetores grandes. Já o Insertion e o Shell são os melhores.
	
	 O Merge, Heap e o Quick mantém uma taxa de tempo parecida em quase todos os tamanhos de vetores.
	
	 Conclui-se que o melhor algoritmo de ordenação para esses casos é o HeapSorte, pois independente do aumento da quantidade de números no vetor, ele não deixa de ser eficiente.
	
	
	
	
	\newpage
	\section{Informações do Computador}	
	
	
	\begin{figure}[h] %\h   significa here = aqui, e [ht] para se a imagem não couber no resto da folha
		\centering
		\includegraphics[scale=1.0]{infopc.jpeg} 
		\caption{Informações do computador} %\Legenda %\para figuras
	\end{figure}%\

\end{document}

